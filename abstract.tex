\thispagestyle{plain}

\begin{flushleft}
\textbf{\Huge Abstract}
\vspace{8mm}
\end{flushleft}

Material acquisition describes the process of inferring properties of materials from observations. We are interested in acquiring a scattering property of materials called the scattering phase function which governs the spherical directionality of scattered light intensity. The phase function largely influences the translucent appearance of materials and is driven by material properties such as the type and size of particles in the medium, so it is essential for characterization.

One approach relies on reducing sample optical density such that light paths through the material are scattered once on average. The phase function is then inferred from the relative scattered intensities in different directions. However, this method is limited to classes of materials that can be sliced thinly or diluted. An alternative approach estimates the scattering phase function by inverting the radiative transfer equation. Although it does not require isolating single-scattering events, it is costly and relies on good initialization due to it use of stochastic gradient descent. Efficient, closed-form approaches that rely on the memory effect have been developed for material acquisition from thick samples. By illuminating a sample with coherent laser light and capturing speckle patterns, correlations within the memory effect range allow the single-scattered component to be measured in the presence of high-order scattering. However, the proposed acquisition system measures the scattering phase function over a small angular range of a few degrees due to inherent angular limits of a 4f system.

In this work, we detail material acquisition over angular ranges approaching $180^\circ$ via scatterometry. Our approach is similar to reflectometry wherein we use two sources fixed to a goniometer that rotates around a sample. We use two mutually coherent laser beams separated by a small angle to maximize speckle correlation. Their respective speckle patterns are acquired using a camera, and the phase function in the mean direction of the two laser beams is proportional to the speckle correlation. The goniometer then rotates the illuminators about the sample to measure the scattering phase function over a large range without the angular limitations inherent in 4f imaging systems. Our results are relevant to graphics applications such as photorealistic augmented reality as well as areas outside graphics and vision such as non-invasive medical diagnostics, remote sensing, and particle sizing for quality assurance.