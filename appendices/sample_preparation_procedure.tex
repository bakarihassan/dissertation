\section{Sample Preparation}

Equipment Required
\begin{itemize}
    \item Microscope slides (two per sample)
    \item Dropper or syringe with capacity greater than or equal to the target sample volume
    \item Ethanol
    \item Isopropyl alcohol
    \item Non-abrasive working surface to prevent scratching slides
\end{itemize}

\begin{tabular}{ l l l l l }
    \toprule[2pt]
    \textbf{Part No.} & \textbf{Vendor} & \textbf{Qty.} & \textbf{Description} \\
    \midrule[0.75pt]
    30392080 & Ohaus & 1 & Dry Block Tube Heater \\
     
    30400154 & Ohaus & 1 & Module Block 20mm 8 Wells \\

    30400193 & Ohaus & 1 & Module Block 50, 15, 1.5mL \\
     
    FA10006M & Gilson & 1 & P1000L 100-1000uL pipette \\
     
    AB0576 & Thermo Fisher & 1 & 25uL gene frame \\

    AB0577 & Thermo Fisher & 1 & 65uL gene frame \\

    AB0578 & Thermo Fisher & 1 & 125uL gene frame \\
     
    GEMINI-20-BLK & American Weigh Scales & 1 & Milligram scale + boat \\
     
    F167014 & Gilson & 1 & D1000 tip reload pack \\

    4916345 & Scientific Labwares & 1 & oval lab spoon \\

    CHWB 1020B & Eisco & 1 & Water squeeze bottle \\

    CHWB 1030 & Eisco & 1 & IPA squeeze bottle \\

    CHWB 1037 & Eisco & 1 & Ethanol squeeze bottle \\

    A20090-50.0 & RPI & 1 & Agarose, 50 G \\

    ??? & Swift & 1 & Microscope \\

    EC 1.3MP & Swift & 1 & Microscope camera \\

    W5-4 & Fisher Chemical & 1 & 4L HPLC Water \\

    904341-2G & Millipore Sigma & 1 & Silica monodisperse 3um spheres \\

    34155 & Kimtech & 1 & Kimwipes \\

    MPR-50504 & Med Pride & 1 & Nitrile gloves \\

    55105 & SPL Life Sciences & 1 & 5mL centrifuge snap tubes \\

    50215 & SPL Life Sciences & 1 & 15mL centrifuge snap tubes \\

    ??? & & 1 & Microfiber towel \\
    \bottomrule[2pt]
\end{tabular}

\begin{enumerate}
    \item Clean all labware with IPA and dry with Kimtech wipes. Clean slides with IPA and dry with optical cleaning cloth.
    
    \item Pour 5 mL HPLC water in a 15 mL tube using a squeeze bottle.

    \item Set the tube heater to 95$^\circ$C and place the 15 mL tube in a heating block.

    \item Once the heater has reach the set point, place a clean 5 mL tube in a heating block. Let both tubes remain in the block for 5 minutes.

    \item Cut \_ mm off the end of a clean pipette tip so its opening is 2 mm in diameter.

    \item Transfer heated HPLC water from the 15 mL tube to the 5 mL tube.

    \item Measure \_ mg agarose powder in a clean weighting boat and add it to the 5 mL tube. Place it in a heating block for 10 minutes, using bottoms-up agitation every 2 minutes.

    \item While waiting, measure \_ mg silica beads in the weigh boat and set aside.

    \item Attach a gene frame to a microscope slide.

    \item Once the water has been heated for 10 minutes, place the slide on a heating block. Then add the silica beads to the agarose solution and let the suspension heat for 5 minutes, using bottoms-up agitation every minute.

    \item Using a 1 mL pipette, cycle the liquid within the 5 mL tube several times.

    \item Ensure the pipette's volume is set to 96\% of the gene frame's volume.

    Note: The following sequence should occur quickly. Otherwise the suspension will cool and begin gelling, reducing the sample quality:

    \item Using heat-resistant gloves, remove the heated slide and place it on a microfiber towel.

    \item Intake the desired volume of the suspension and aliquot it while moving the pipette tip across the area of the gene frame. This will minimize the height of the liquid and will prevent overflow when placing the slide cover on the gene frame.

    \item Using another microscope slides, attach the slide cover from one side to another, lengthwise along the gene frame.

    \item Press the slide flat against the slide cover and leave it in place to ensure the cover is parallel with the surface of the sample slide.
    
    \item Weigh-out the amount using a scooper and clean weigh boat
    
    \item Combine microspheres with \_ mL ethanol in a centrifuge tube
        \begin{itemize}
            \item Micro sphere volume: $14.137 \mu m^3 = 1.4137 \times 10^{-11} cm^3$
            \item Micro spheres per gram: $3.1 \times 10^{11} \frac{sphere}{g} = \frac{1 \; cm^3}{2.196 \; g} \big(\frac{10^4 \; \mu m}{1 \; cm} \big)^3 \frac{1}{(\frac{4}{3}\pi 1.5^3) \; \mu m^3}$
            \item Ethanol Volume: $\frac{ 3.1 \times 10^{11} \; spheres}{1 \; g} \cdot \frac{1.4137 \times 10^{-11} cm^3}{1 \; sphere} \cdot \frac{1 \; mL}{1 \; cm^3} \cdot 125 \cdot x \; grams$
        \end{itemize}
        \item Measure-out 0.1 g of agarose powder and place in second tube
        \item Fill second tube with 10 mL HPCL water
        \item Fill 10 tubes with HPCL water
        \item Insert all 12 tubes in heating block
        \item Set the mixer to \_ degrees Celsius and begin heating the tubes.
        \item After \_ minutes, start mixing the tubes at \_ RPM for \_ minutes.
        \item Once done, pour the water tubes into a TBD container and place the slides in the water.
        \item Pour the microsphere-Ethanol mixture into the agarose mixture and resume mixing and heating for \_ minutes (depending on time required to heat-up slides).
        \item Remove slides, dry them, attach gene frames, and place on non-abrasive surface. The next few steps should be completed quickly to avoid the slides cooling too much.
        \item When mixing is complete, remove tube and agitate 20 times using the bottoms-up method
        \item Place the mixture back in the mixer and use a pipette to transfer liquid to the center of the gene frame. The amount transferred should match the volume of the gene frame.
        \item Sandwich the sample using the top face of another microscope slide (not the side that was facing down on the table). While sandwiching, touch the slides as far away from the gene frame as possible, and on the edges.
        \item Place on the rotator at \_ RPM for \_ minutes (need to calculate cooling rate). The liquid and slides should now be at a uniform temperature.
        \item Inspect the sample using a microscope objective to determine the uniformity. Need to develop procedure for this. Maybe 8 locations (3 along top, 3 along bottom, two intermediate positions along sample center line). Then compute variance of number of particles visible across images? Use that as a benchmark.
        \item Clean pipette internals, glassware using IPA
        
\end{enumerate}