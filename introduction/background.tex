\section{Background}

Scattering refers to the behavior of light when it interacts with a medium containing particles or when light interacts with the interface between two mediums of different properties. When light interacts with a particle in a scattering medium, the scattering event generates many additional light paths that all undergo additional scattering events before arriving at the observer. Scattering is essential to the appearance of food, liquids, skin, and other translucent materials, and an understanding of scattering is critical to determining their appearances. Under incoherent illumination, scattering produces smooth highlights with gradual falloff from the area of illumination. However, under coherent illumination, scattering produces speckle. Although speckle may appear random, it has a strong structure determined by the properties of the illumination and the scattering medium.

\begin{itemize}
    \item Here's why we can reduce that to SS
    \item Once we have SS, this is how we make measurements
\end{itemize}

\subsection{Scattering material representation}
Scattering materials are generally composed of small particles with varying refractive properties we describe through the bulk statistical properties of the material. We use three statistical properties to parameterize the scattering material. The extinction coefficient describes the extinction cross-section of the scattering particles per unit volume. It is therefore proportional to the density of scattering particles inside the material. The extinction coefficient is the sum of the absorption and scattering coefficients $\sigma_t = \sigma_a + \sigma_s$ which represent, respectively, the portion of light absorbed and scattered per unit length along the path. The material's phase function $\rho(\arccos(\uvec{i} \cdot \uvec{v}))$ describes the directionality of scattered light and determines the portion of light scattered towards direction $\uvec{v}$ when a scatterer is illuminated from direction $\uvec{i}$. A scatterer's phase function is dictated by its shape and refractive index. Phase functions for spherical particles can be computed analytically using Mie theory\cite{bohren2008absorption, frisvad2007computing, hulst1981light}. $\rho$ is generally assumed to be isotropic. This means its value depends only on the inner product of the illumination and viewing directions, and not on the absolute directions. This may be relaxed by adding an anisotropy parameter $-1 \leq g \leq 1$ where $g=-1$ corresponds to fully backward scattering, $g=0$ means light is scattered equally in all directions, and $g=1$ is full forward scattering. The mean free path (MFP) of a material is defined as the average distance light travels inside the volume between two successive scattering events. The MFP is the inverse of the extinction coefficient $MFP=1/\sigma_t$. When working with scattering volumes, it is common to express its geometric dimensions with respect to the MFP. For example, a volume with optical depth $OD = 4$ means its thickness is $4 \cdot MFP$. This means that light traveling through the medium is scattered four times in average.

Our work is primarily interested in the phase function $\rho$ and will not discuss scattering coefficients. Our work seeks to acquire $\rho$ as a general function and does not assume common parameterizations such as the Henyey-Greenstein phase function \cite{henyey1940diffuse}.

\begin{equation}
    p_{HG}(\theta) = \frac{1}{4\pi} \frac{1 - g^2}{(1 + g^2 - 2 g \cos{\theta})^{3/2}}
\end{equation}

\subsection{Phase function from speckle images}

\paragraph{Validating extended range phase functions}
Phase functions can be validated by comparing correlations to results computed from full Monte-Carlo simulations. Monodispersions of microscopic silica beads are well-suited for validation because their scattering effects are well described by Mie theory. Alterman et al. validate their results by comparing closed-form correlations from Equation \_\_\_ to results to a Monte-Carlo simulator \cite{bar2019monte} that has been verified against an accurate wave solver \cite{thierry2015mu}. We assess our results in two. First, we verify our acquisition setup and single-scattering computations by comparing acquired $3\mu m$ and $10\mu m$ monodispersions to Mie theory. We then validate results for non-monodisperse samples that are not easily characterized for simulation (e.g., mustard, milk, honey) by comparing our acquired phase functions to those acquired by Alterman et al. over angular ranges up to approximately 8 degrees \cite{alterman2022direct}. Given extended-range verification against theory and limited-range validation with related work, we consider our extended-range measurements valid.

\paragraph{Phase function from single scattering models}
A simple method for measuring phase functions is acquiring optically thin samples that scatter light once on average ($OD \approx 1$). This method is as simple as reflectometry: we illuminate a sample in direction $\vec{i}$ and measure the light received in direction $\vec{v}$, and the phase function is the portion of energy corresponding to scattering angle $\arccos(\uvec{i} \cdot \uvec{v})$. In the paraxial regime, we can apply the small-angle approximation to equate the scattering angle as the norm of the displacement vector between the illuminating and viewing directions $\vec{\tau} = \uvec{v} - \uvec{i}$, and $ |\vec{\tau}| = \arccos(\uvec{i} \cdot \uvec{v})$. This method fails with increasing material thickness due to multiple scattering.
