\begin{itemize}
    \item Scattering materials are ubiquitous, important in RS, medical, etc.
    \item All critical applications for human beings benefit from deeper understanding of optical property of scattering
    \item The nature of these materials makes this challenging
    \item As a result, there have been many approaches (reiterate some related work at high level)
    \item Goal of this thesis is to combine both approaches through direct observation based on correlations.
    \item We improve by having a system that can do this over wide angles with direct and accurate measurements
    \item Statement of contributions (design and calibrations discussed in sections blah, blah, blah)
\end{itemize}

Scattering


Scattering refers to the propagation in media containing small, discrete scatterers with varying refractive properties. As light propagates through the medium, it interacts with multiple scatterers along its path, and each interaction changes its direction of travel. Scattering is common for a large variety of materials including biological tissues, the atmosphere, liquids, and cosmetics.

In this work, we increase the limited angular acquisition range from $8^\circ$ to greater than $100^\circ$ for a broader class of materials by combining the benefits of scatterometry and closed-form phase function computation from speckle correlations. Our scatterometer consists of two mutually coherent sources separated by a fixed, small elevation angle. Each beam illuminates a scattering medium, and scattered light is captured by a camera aperture. The image of each beam's scattered light is a speckle image, and both speckle images are correlated to find the single-scattered component proportional to the phase function. Because the medium scatters light spherically, we can capture speckle images with the beam pair located arbitrarily. By rotating the beam pair azimuthally with the sample at the center of rotation, we can measure speckle correlation over a broad range of scattering angles.