Scattering materials are ubiquitous, and they play key roles in fields such as tissue and blood analysis in medical imaging used to classify tumors as malignant or benign, produce characterization in the agriculture industry, soil analysis and material identification in remote sensing, and particle size estimates for quality assurance and shelf life estimates of cosmetics. These applications have widespread societal impacts, and they benefit from a deeper understanding of the optical properties of scattering. However, the complex nature scattering poses challenges for advancements in end applications. As a result, there has been a variety of approaches to understanding the underlying scattering phenomena. One class of approaches focuses on optically thick materials whose analysis can be approximated using the diffusion approximation. Monte-Carlo rendering has been used to consider materials of arbitrary thickness. However, they are computationally expensive and require proper initialization. The final class works under the single-scattering approximation and focuses on materials where single-scattering can be isolated. One way to isolate single-scattering is to dilute materials so light scatters once on average. This allows simple characterization methods similar to reflectometry for a narrow class of materials. Single-scattering can also be isolated for multi-scattering materials by using coherent illumination to compute scattering statistics dominated by single-scattered light. However, these methods have been limited to small scattering angles. The goal of this thesis is to combine the advantages of both approaches under the single-scattering approximation. We compute correlations to isolate single-scattered light in multi-scattering materials, and we expand the range of scattering angles from $8^\circ$ to greater than $100^\circ$ using scatteometry to make direct and accurate measurements for a broad class of materials.

\subsection{Thesis contributions}
In this thesis, we introduce a novel scatterometry system for material acquisition over a broad range of scattering angles exceeding $100^\circ$. We are particularly interested in opaque materials that exhibit multi-scattering. This thesis establishes the design approach for scatterometry under the single-scattering approximation as well as the calibration and alignment processes. The high-level design and unique aspects are detailed in Section \_, the calibration and alignment processes are covered in Sections \_, and results are discussed in Section \_.

% In this work, we increase the limited angular acquisition range from $8^\circ$ to greater than $100^\circ$ for a broader class of materials by combining the benefits of scatterometry and closed-form phase function computation from speckle correlations. Our scatterometer consists of two mutually coherent sources separated by a fixed, small elevation angle. Each beam illuminates a scattering medium, and scattered light is captured by a camera aperture. The image of each beam's scattered light is a speckle image, and both speckle images are correlated to find the single-scattered component proportional to the phase function. Because the medium scatters light spherically, we can capture speckle images with the beam pair located arbitrarily. By rotating the beam pair azimuthally with the sample at the center of rotation, we can measure speckle correlation over a broad range of scattering angles.