\section{Speckle Correlations}

Scattering refers to the propagation in media containing small, discrete scatterers with varying refractive properties. As light propagates through the medium, it interacts with multiple scatterers along its path, and each interaction changes its direction of travel. Scattering is common for a large variety of materials including biological tissues, the atmosphere, liquids, and cosmetics.

In this work, we increase the limited angular acquisition range from $8^\circ$ to greater than $100^\circ$ for a broader class of materials by combining the benefits of scatterometry and closed-form phase function computation from speckle correlations. Our scatterometer consists of two mutually coherent sources separated by a fixed, small elevation angle. Each beam illuminates a scattering medium, and scattered light is captured by a camera aperture. The image of each beam's scattered light is a speckle image, and both speckle images are correlated to find the single-scattered component proportional to the phase function. Because the medium scatters light spherically, we can capture speckle images with the beam pair located arbitrarily. By rotating the beam pair azimuthally with the sample at the center of rotation, we can measure speckle correlation over a broad range of scattering angles.

\section{Related work} % Background on material acquisition

Material acquisition is the task of recovering the intrinsic properties of materials based on their appearance. It is of great importance in many applications. For example, tumors can be detected and classified as malignant or non-malignant \cite{boas2001imaging}; important blood properties such as red and white blood counts can also be analyzed \cite{berne2000dynamic, durduran2010diffuse}; in materials science, material acquisition is used to validate the fidelity and shelf life of material samples \cite{sumin2019geometry}; and the chemical compositions of nanodispersions can be inferred for particle sizing applications \cite{pine1990diffusing}.

Inverse radiative transport \cite{bal2009inverse} is studied heavily in graphics as well as the physical and biomedical sciences. While inverse radiative transport methods for scattering media fall into three main categories, methods using the \textit{diffusion} approximation focus on optically thick media where high-order scattering is dominant. While this approximation simplifies inference and is suitable for both homogeneous and heterogeneous materials \cite{farrell1992diffusion, munoz2011bssrdf, papas2013fabricating}, it introduces parameter ambiguities. Similarity relations are hierarchical parameter relationships that allow scattering parameters to be altered without significantly altering the medium's spatial properties. These relations can be derived from transport equations to accelerate Monte Carlo simulations \cite{wyman1989similarity}. However, a radiance field computed via Monte Carlo simulations can be described by multiple, distinct sets of parameters, and finding mulutiple candidate solution sets is generally challenging. Parameter space warping and exploiting similarity relations have improved the efficiency of iterative solvers \cite{zhao2014high}.

Rather than focusing purely on high-order scattering, another class of methods considers all paths of arbitrary lengths. Given a set of input images, they estimate material parameters whose combinations closely match the inputs when simulated using Monte-Carlo rendering \cite{dutre2018advanced, novak2018monte}. Differentiable rendering determines the effects of changes in scattering parameters by estimating derivatives of images. Traditionally, these estimtes have been approximate models that ignore complex light transport effects such as subsurface scattering and inter-reflections \cite{loper2014opendr}. Differentiable Monte Carlo rendering overcomes these limiations by computing derivatives while accounting for all light transport effects \cite{gkioulekas2016evaluation, gkioulekas2013inverse, khungurn2015matching, nimier2020radiative, nimier2019mitsuba}. Machine learning approaches offer lower computational complexity at the cost of reduced robustness and diminished physically accurate solutions. Encoder networks can be paired with Monte Carlo renderers to improve their generalization to scenes with unseen geometry and light sources \cite{wu2017neural, che2020towards}. Energy losses in neural radiance fields are mitigated by efficient indirect illumination estimation via spherical harmonics \cite{zheng2021neural}. While these approach are more general in nature and can handle arbitrarily thick materials, they are computationally expensive and require proper initialization.

The final class of methods are based on the \textit{single scattering} approximation. The first approach assumes the medium is thin enough optically such that photons only scatter once when traveling through the medium. Since the scattering phase function is defined in terms of single-scattering, this allows the phase function to be observed directly. Although this method is as simple, it is limited to a narrow classes of materials such as gases and liquids of low viscosity \cite{narasimhan2006acquiring}. Viscous liquids and thin solids can be acquired by illuminating materials with coherent light and computing the correlations of speckle images. Speckle image correlations are dominated by single-scattered light, and \cite{alterman2022direct} showed that the phase function is proportional to the square root of the correlation and can be computed using a closed-form equation. However, this method is limited to measuring phase functions up to $8^\circ$ due to aberrations.
