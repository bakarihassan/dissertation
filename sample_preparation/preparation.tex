\subsection{Equipment}
\begin{itemize}
    \item Microscope slide - Menzel Glaser SUPERFROST Frosted End (no. 12-550-15)
    \item Cover glass - Corning 24x50 mm (no. 2975-245)
    \item Gene frame 125 $\mu$L - Thermo Scientific (no. AB0578)
    \item Low melt temperature agarose - RPI (no. 9012-36-6)
    \item DI(?) water
    \item 10$\mu$m silica monodisperse, non-porous - Sigma-Aldrich (no. 904368)
    \item 15 mL centrifuge tube
    \item Pipette set to 125 $\mu$L intake
    \item Isopropyl alcohol
    \item Centrifuge tube heater - Ohaus 2 block dry block heater (no. HB2DG)
    \item mg Scale
\end{itemize}

\subsection{Procedure}
\begin{enumerate}
    \item Pour 5 mL DI(?) water into a 15 mL centrifuge tube (tube #1) and a 10 mL tube (tube #2)
    \item Cut a clen pipette tip so its opening is 2-3mm in diameter.
    \item Place the pipette tip inside tube #2. Its wide tip should be above the water's surface.
    \item Heat both tubes to 100 $^\circ$C
    \item While the tubes are heating, place the microscope slide on the border of the heater
    \item Create 1\% agarose solution by adding 50 mg agarose powder to centrifuge tube #1 (see Equation)
    \item Shake centrifuge tube 1 periodically until agarose is fully dissolved
    \item For an OD 1 sample, add \_\_\_ mg silica monodisperse to tube #1(see Equation)
    % $C_{SiO_2} = \frac{N_{SiO_2}}{V_{ag}}$ % Number density
    % $m_{\circ} = \rho_{SiO_2} * V_{\circ}$ % Mass of a single bead in grams
    % $D_{\circ}$ % Bead diameter
    % $V_{ag} % Volume of agarose suspension$
    % $L$ % Sample thickness
    % $A$ % Sample cross-section area (gene frame area assuming it is full)
    % $MFP = \frac{1}{C * A}$
    \item Heat the solution for 5 minutes, shaking periodically.
    \item While the solution is heating, carefully remove the hot microscope slide from the heater. Remove one of the gene frame laminates, and adhere the gene frame on the slide so it is centered vertically and horizontally. The warm slide improves gene frame adhesion and promotes a better seal.
    \item Place the slide and the cover glass in the center of the work area. Let the slide cool to room temperature. We found that depositing liquid onto a hot slide increases bead mobility and causes them to fall to the bottom of the gene frame.
    \item Attach the heated pipette tip in tube #2 to the pipette.
    \item With the centrifuge in the heating block, place the pipette tip at the bottom of the centrifuge tube and use the plunger to repeatedly cycle liquid in and out of the pipette tube. The turbulence disturbs beads that have fallen to the bottom of the tube and creates a more uniform concentration.
    \item Slightly raise the tip from the bottom to avoid an air seal, and slowly press the plunger and slowly release it as you raise the tip from the bottom of the tube to the surface of the liquid. The plunger should be fully released before the tip reaches the surface to ensure there is no air intake.
    \item In quick succession:
    \item \begin{enumerate}
        \item Place the pipette tip against the surface of the microscope slide, and dispense the liquid into the gene frame by pressing the plunger to its second stop. Remove the pipette from the liquid before releasing the plunger.
        \item Apply the cover glass from one of its long ends to the other to avoid air bubbles. It may be helpful to slide a smooth, small cylindrical bottle across the cover glass as you apply it to provide additional pressure.
    \end{enumerate}
\end{enumerate}

\subsection{Sample Characterization}
